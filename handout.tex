	\documentclass{article}
	
	\usepackage{mathtools}
	\usepackage{amsfonts}
	
	\title{Handout: SHA 1-3}
	\author{Deniz Kunze}
	
	
	\renewcommand*\contentsname{Inhalte}
	
	\begin{document}
	\maketitle
	\newpage
	\tableofcontents
	\newpage

	\section{Motivation für Hashing im Allgemeinen}
	\subsection{Begriffsklärungen}
	\subsubsection{Kryptographie}
	\subsubsection{Authentifikation}
	\subsubsection{Zertifizierung}
	\subsubsection{infeasable (TODO)}
	\subsection{Asymetrische Verschlüsselung}
	\subsection{Wie funktionieren Authentifizierung und Zertifizierung?}
	\subsection{Probleme von RSA}
	\section{Anforderungen an einen Secure-Hashing-Algorithmus}
	\section{SHA-1}
	\section{SHA-2}
	\section{SHA-3}
	\section{SHA 1-3 heute}
	\section{Fazit}
	\section{	Quellen}


\end{document}